\documentclass[letterpaper,12pt]{article}

% works with pdfLaTeX
\usepackage{type1cm} % scalable fonts
\usepackage{lettrine}
\usepackage{graphicx}
\usepackage{pdflscape}
\usepackage{float}
\usepackage{caption}
\usepackage{rotating}
\usepackage[margin=0.75in, top=0.5in, bottom=0.5in]{geometry}


\begin{document}
\pagenumbering{gobble} 
\title{Network Performance Analysis}
\author{Automated Network Monitor v0.1}
\date{\today}
\maketitle

\section{Internet Service Performance}

\lettrine{T}{his}
report is generated by a device that is designed to monitor a SOHO (Small Office Home Office) network, including the Internet connection and the local WiFi and Lan environments. By default, the report includes data from the moment the report is run reaching back to midnight at the start the day the report is run.  Optionally, it can reach back to a midnight-to-midnight view of the prior day. This is most useful for automated runs.\\

The device measures traffic between the ISP provided connection (e.g. cable or dsl modem) and the customer provided router.  It uses a 'transparent bridge' and therefore does *not* require configuration or enrollment with your ISP.  In addition the device runs periodic measurements that detect whether the internet connection is fundamentally 'Up' or 'Down', measure download and upload speed directly from the main router, measure speeds reaching across the WiFi 2.4Ghz and 5Ghz networks, measure DNS resolution, and more. Each of these test are run at intervals appropriate to the test, and then summarized for reporting here. \\

All data collected is kept in a database.  This allows the report to be run or re-run at any time. By default, records older than 14 days are discarded. This keeps the database from growing out of control. \\

\newpage
\begin{landscape}
\section{Basic Up/Down}
\lettrine{B}{asic}
Up/Down is monitored by pinging Google Name Servers every 59 seconds.  Two pings are issued and if both are missed that is logged. These records are summarized on a 10 minute basis, and if two minor intervals show down, the 10 minute interval is marked down. 
\begin{center}
\makebox[\textwidth]{\includegraphics[width=1.5\textwidth]{dailyReportPing.png}}
\captionof{figure}{Internet Up/Down}\label{visina8}%      only if needed  
\end{center}
\end{landscape}

\newpage
\begin{landscape}
\section{Basic Internet Traffic}
\lettrine{T}{hese}
charts shows the traffic flowing across the internet link.  It is measured by a probe inserted between the main router and the cable/dsl modem.  It shows all traffic, including tests generated by this very monitor, all web browsing, streaming, and more. Both charts should always mirror each other's tx and rx; if they are not,
insvestigation of the cause is advised. 
\begin{center}
%\makebox[\textwidth]{\includegraphics[width=1.5\textwidth]{dailyReportIntfbr0.png}}
\makebox[\textwidth]{\includegraphics[width=1.5\textwidth]{dailyReportIntfeth1.png}}
\makebox[\textwidth]{\includegraphics[width=1.5\textwidth]{dailyReportIntfeth2.png}}
\captionof{figure}{Overall Internet traffic.}\label{visina8}%      only if needed  
\end{center}
\end{landscape}

\newpage
\begin{landscape}
\section{Download and Upload Speeds}

\lettrine{T}{his}
chart plots Internet speedtest results, including the download and upload speeds in megabits per second (Mbps). These tests are run in parallel with any other Internet communication that may be occurring on the network. Network activity such as large downloads or media streaming sessions will affect these test results.
\begin{center}
\makebox[\textwidth]{\includegraphics[width=1.5\textwidth]{dailyReportSpeedTest.png}}
\captionof{figure}{Speedtest}\label{visina8}%      only if needed  
\end{center}
\end{landscape}


\newpage
\begin{landscape}
\section{WiFi Speeds}

\lettrine{T}{his}
chart plots Iperf3 results that show the speeds achievable on the WiFi networks inside this building.  To stress: WiFi speeds have absolutely nothing to do with Internet speeds... except that WiFi is the "last link" to Phones and Tablets in this building.  Therefore, if WiFi has problems, it is often perceived as "the Internet is slow".  
\begin{center}
\makebox[\textwidth]{\includegraphics[width=1.5\textwidth]{dailyReportIperf2.png}}
\captionof{figure}{2.4Ghz Iperf3 Measured Speeds}\label{visina8}%      only if needed  
\makebox[\textwidth]{\includegraphics[width=1.5\textwidth]{dailyReportIperf5.png}}
\captionof{figure}{  5Ghz Iperf3 Measured Speeds}\label{visina8}%      only if needed  
\end{center}
\end{landscape}

\newpage
\begin{landscape}
\section{Domain Name Resolution Speeds}

\lettrine{T}{his}
chart plots 'dig' results that show the number of milliseconds necessary to translate a name, such as www.google.com, to the address of that server.  To stress: DNS speeds have absolutely nothing to do with Internet speeds... except that DNS resolution is the very first step of opening a web page.  Therefore, if DNS resolution has problems, it is often perceived as "the Internet is slow".  
\begin{center}
\makebox[\textwidth]{\includegraphics[width=1.5\textwidth]{dailyReportDig.png}}
\captionof{figure}{Local and Google DNS Resolution}\label{visina8}%      only if needed    
\end{center}
\end{landscape}

\end{document}

